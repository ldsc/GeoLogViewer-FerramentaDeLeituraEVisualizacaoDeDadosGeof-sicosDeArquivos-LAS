%% LyX 2.4.3 created this file.  For more info, see https://www.lyx.org/.
%% Do not edit unless you really know what you are doing.
\documentclass[12pt,oneside,brazil,normaltoc, sumarioincompleto]{extbook}
\usepackage[T1]{fontenc}
\usepackage[latin9]{inputenc}
\usepackage{babel}
\usepackage{float}
\usepackage{makeidx}
\makeindex
\usepackage{graphicx}
\usepackage[a4paper]{geometry}
\geometry{verbose,tmargin=20mm,bmargin=20mm,lmargin=15mm,rmargin=10mm}
\usepackage{fancyhdr}
\pagestyle{fancy}
\usepackage{setspace}
\onehalfspacing
\usepackage[unicode=true,
 bookmarks=true,bookmarksnumbered=true,bookmarksopen=true,bookmarksopenlevel=1,
 breaklinks=true,pdfborder={0 0 0},pdfborderstyle={},backref=page,colorlinks=false]
 {hyperref}
\hypersetup{pdftitle={Projeto Engenharia: Desenvolvimento Software Aplicado XXXXX},
 pdfauthor={Nome membros equipe; Professor Andr� Duarte Bueno; },
 pdfsubject={Descrever assunto},
 pdfkeywords={Copiar do resumo}}

\makeatletter

%%%%%%%%%%%%%%%%%%%%%%%%%%%%%% LyX specific LaTeX commands.
%% Because html converters don't know tabularnewline
\providecommand{\tabularnewline}{\\}

%%%%%%%%%%%%%%%%%%%%%%%%%%%%%% User specified LaTeX commands.
%-----------------------------------------------------------------
% Para incluir formata��es especificas para apresenta��es
%-----------------------------------------------------------------
%https://tex.stackexchange.com/questions/5894/latex-conditional-expression
% Abaixo inserimos o pacote etoolbox que permite a gest�o de if..else
\usepackage{etoolbox}
% Cria o flag
\newtoggle{FormatoApresentacaoTRUE}
\newtoggle{IncluirBibliografiaNoCapituloTRUE}
%Seta o flag - deixe true se for para gerar apresenta��o de aula e false para formato livro
%\toggletrue{FormatoApresentacaoTRUE}
\togglefalse{FormatoApresentacaoTRUE}
%\toggletrue{IncluirBibliografiaNoCapituloTRUE}
\togglefalse{IncluirBibliografiaNoCapituloTRUE}

% Para usar no meio dos arquivos do lyx crie um comando latex e cole o texto abaixo
%\iftoggle{FormatoApresentacaoTRUE}{..verdadeiro..}{..falso..}
%\iftoggle{FormatoApresentacaoTRUE}{\newpage}{}

%---------------------------------------------------------------
% Para adicionar/excluir uma se��o inteira
%---------------------------------------------------------------
%https://tex.stackexchange.com/questions/193295/lyx-conditional-sections
% Abaixo criamos um novo if
\newif\ifIncluirSecaoProgramacaoAvancada
%\FormatoApresentacaoWidefalse
\IncluirSecaoProgramacaoAvancadatrue
% Para usar
% \ifIncluirSecaoProgramacaoAvancada no in�cio do blobo
% \fi para finalizar o bloco

%---------------------------------------------------------------
%GERAL
%---------------------------------------------------------------
\usepackage{verbatim}
%%\usepackage{algorithm}
\usepackage{fancyvrb}
\usepackage{keyval} 
\usepackage{indentfirst}
%\usepackage{color}
\usepackage{xcolor}
\definecolor{azulclaro}{rgb}{0.6,1,1}%   rgb color model
\definecolor{BLACK}{rgb}{0,0,0}%   rgb color model
\definecolor{BLUE}{rgb}{0,0,1}%   rgb color model

%Informa que vai usar o pacote listings, disponibilizado em /usr/share/texmf/doc/latex/styles/listings.dvi
\usepackage{listings}

%Define um novo comando, chamado lst
%observe que lst chama o comando  lstinputlisting.
\newcommand{\lst}[2]{%
    \noindent\rule[-1ex]{\textwidth}{0.3mm}\vspace{-1ex}
    \lstinputlisting[caption={#2},label={#1},stringspaces=false,frame={tb},lineskip=-1pt,extendedchars=true,%
    basicstyle=\footnotesize\tt,labelstep=1,labelstyle=\tiny,indent=2em,language=Java,breaklines]{#1}
    \vspace{1ex}%
}

%\listfiles
%\usepackage[]{hyperref}

%\usepackage{mathptmx}  % instead of package times

%\usepackage[scaled=0.9]{helvet} % or [scaled=0.92], if you like

%% Pacote de cita��es para formato abnt
%%\usepackage[num]{abntcite}
%%\usepackage[alf]{abntcite}

%%\usepackage[cam,a4,center]{crop}
%% a4 � o tamanho f�sico
%% center � para centralizar
%% cam inclui quatro marcas diferentes

%%\usepackage[cam,width=20truecm,height=28truecm,center]{crop}
%\usepackage[cam,width=18truecm,height=26truecm,center]{crop}

%\noindent

\makeatother

\begin{document}
\begin{center}
{\large UNIVERSIDADE ESTADUAL DO NORTE FLUMINENSE}\\
{\large LABORAT�RIO DE ENGENHARIA E EXPLORA��O DE PETR�LEO}\\
{\large CENTRO DE CI�NCIA E TECNOLOGIA}{\large\par}
\par\end{center}

\begin{center}
{\large}{\large\par}
\par\end{center}

\vspace{3cm}

\begin{center}
{\large PROJETO DE ENGENHARIA}\\
{\large DESENVOLVIMENTO DO SOFTWARE }\\
{\large XXX}\\
{\large DISCIPLINA LEP - 0XXXX: Introdu��o ao Projeto de Engenharia}\\
{\large Setor de Modelagem Matem�tica Computacional}{\large\par}
\par\end{center}

\vspace{3cm}

\begin{center}
{\large Vers�o 1:}\\
{\large AUTORES}\\
{\large Vers�o 2:}\\
{\large AUTORES}\\
{\large Prof. Andr� Duarte Bueno}{\large\par}
\par\end{center}

\vspace{5cm}
\thispagestyle{empty}
\begin{center}
{\large MACA� - RJ}\\
{\large Janeiro - 2023}{\large\par}
\par\end{center}


\tableofcontents{}


\chapter{Introdu��o\label{cha:Introdu=0000E7=0000E3o}}

\fancyhead[L]{Desenvolvimento do Projeto de Engenharia - Projeto Detalhado}
\fancyhead[C]{}
\fancyhead[R]{\thepage}
\fancyfoot[L]{NomeProfessor-Cliente}
\fancyfoot[C]{}
\fancyfoot[R]{\today}
\pagenumbering{arabic}

Apresenta-se aqui a proposta de desenvolvimento do software XX\_nome\_XX.
Um software aplicado a engenharia de petr�leo e que visa ...descrever.
em um par�grafo super resumido.


\section{Identifica��o da Proposta}

\subsubsection{N�mero da proposta}
\begin{itemize}
\item LDSC-2023-1-P50
\end{itemize}

\subsubsection{Tipo de investimento /divulga��o}
\begin{itemize}
\item PROJETO DE PESQUISA E DESENVOLVIMENTO / DESENVOLVIMENTO DE SOFTWARE
- Vers�o 2 
\end{itemize}

\subsubsection{Tipo de instrumento contratual}
\begin{itemize}
\item Trabalho de disciplina
\end{itemize}

\section{Identifica��o do Projeto }

\subsubsection{T�tulo do projeto }
\begin{itemize}
\item \textquotedbl T�tulo do projeto aqui\textquotedbl{}
\end{itemize}

\subsubsection{Palavras-chave}
\begin{itemize}
\item ...coloque aqui lista de palavras chaves - at� 5...
\end{itemize}

\section{Identifica��o da Universidade, Institui��es e Empresas Participantes}

\subsection{Universidade}
\begin{itemize}
\item UNIVERSIDADE ESTADUAL DO NORTE FLUMINENSE DARCY RIBEIRO/UENF
\item CENTRO DE CI�NCIA E TECNOLOGIA - CCT
\item DEPARTAMENTO DE ENGENHARIA DE PETR�LEO - LENEP
\item SETOR DE MODELAGEM MATEM�TICA COMPUTACIONAL
\item Representante pela universidade:
\begin{itemize}
\item Professor(a): Nome/email/telefone.
\end{itemize}
\end{itemize}

\subsection{Institui��o/Funda��o}
\begin{itemize}
\item Nome institui��o se houver.
\item Representante da institui��o/funda��o:
\begin{itemize}
\item Professor(a): Nome/email/telefone.
\end{itemize}
\end{itemize}

\subsection{Empresa}
\begin{itemize}
\item Nome empresa se houver.
\item Dados da empresa (CNPj, contatos).
\item Representante pela empresa:
\begin{itemize}
\item Engenheiro(a): Nome/email/telefone.
\end{itemize}
\end{itemize}

\subsection{Equipe}
\begin{itemize}
\item Nome aluno/email/telefone.
\item Nome aluno/email/telefone.
\item Nome aluno/email/telefone.
\item Representante pela equipe:
\begin{itemize}
\item Estudante: Nome/email/telefone.
\end{itemize}
\end{itemize}

\section{Resumo\label{sec:Resumo}}
\begin{itemize}
\item Limite de 500 palavras, ent�o seja bem direto!
\item Escreva no final, depois de escrever todo o resto.
\end{itemize}

\section{Escopo do Problema \label{sec:Escopo-do-Problema}}
\begin{itemize}
\item Definir o escopo do problema de engenharia\footnote{\begin{itemize}
\item problemas vinculados �: i) engenharia de petr�leo, ii) algoritmos
computacionais, iii) sistemas de software
\end{itemize}
}, a ideia geral. Destacar sua import�ncia/relev�ncia.
\item Definir a abrang�ncia, delimintando o assunto (para que caiba no prazo
de at� 24 meses). 
\item Situ�-lo no tempo e no espa�o. Situ�-lo em rela��o a outros concorrentes
(sistemas similares, como softwares similares).
\end{itemize}

\section{Objetivos\label{sec:Objetivos}}

Os objetivos deste projeto de engenharia s�o:
\begin{itemize}
\item Objetivo geral:

\begin{itemize}
\item Descreva aqui o objetivo geral do projeto de engenharia, incluindo
v�nculos com engenharia de petr�leo e com modelagem matem�tica computacional
(ideia de l�gica, algoritmos,...).
\item Desenvolver um projeto de engenharia de software para ...{[}.....descrever
de forma clara, direta, objetiva, o objetivo geral do software{]}.
\end{itemize}
\item Objetivos espec�ficos:

\begin{itemize}
\item Modelar f�sica e matematicamente o problema.
\item Modelagem est�tica do software (diagramas de caso de uso, de pacotes,
de classes).
\item Modelagem din�mica do software (desenvolver algoritmos e diagramas
exemplificando os fluxos de processamento).
\item Calcular XXX{[}.....descrever de forma clara, direta, objetiva, cada
objetivo espec�fico, cada parte do software{]}.
\item Calcular XXX{[}.....descrever de forma clara, direta, objetiva, cada
objetivo espec�fico, cada parte do software{]}.
\item Simular (realizar simula��es para teste do software desenvolvido).
\item Implementar manual simplificado de uso do software.
\item 
\end{itemize}
\end{itemize}




\section{Justificativas\label{sec:Justificativas}}
\begin{itemize}
\item Colocar as justificativas para o desenvolvimento da solu��o de engenharia
(produto, processo, softwares ou sistema a ser desenvolvido).
\item Apresentar usos e aplica��es em engenharia.
\end{itemize}

\section{Resultados Esperados\label{sec:Resultados-Esperados}}
\begin{itemize}
\item Um simulador de engenharia com interface amig�vel, com manuais t�cnico
cient�ticos, aplicado � .
\end{itemize}

\section{Metodologia\label{sec:Metodologia}}
\begin{itemize}
\item Descrever a metodologia da solu��o do problema te�rico/conceitual..
ou seja, metodologia para solu��o do problema de engenharia...
\item A Figura \ref{fig: Metodologia utilizada no desenvolvimento do sistema}
apresenta a metodologia a ser utilizada no desenvolvimento do sistema.
\begin{figure}[h]
\begin{centering}
\includegraphics[width=1\textwidth,keepaspectratio,height=0.8\textheight]{../../imagens/ProjetoEngenharia-Etapas-Geral-wide}
\par\end{centering}
\caption{Metodologia utilizada no desenvolvimento do sistema\label{fig: Metodologia utilizada no desenvolvimento do sistema}}
\end{figure}
\end{itemize}

\section{Mecanismos de Acompanhamento da Execu��o\label{sec:Mecanismos-de-Acompanhamento}}
\begin{itemize}
\item Para o acompanhamento da execu��o do projeto iremos usar a metodologia
SCRUM com as atividades do projeto hospedadas no site \emph{github/projects
}(ou equivalente como o trello).
\item Reuni�es mensais com os clientes.
\item Relat�rios no final do semestre, com apresenta��o oral.
\end{itemize}

\section{Informa��es Adicionais Espec�ficas\label{sec:Informa=0000E7=0000F5es-Adicionais-Espec=0000EDficas}}
\begin{itemize}
\item Coloque aqui informa��es adicionais, importantes, relevantes e que
n�o se encaixaram nos demais t�picos deste documento.
\end{itemize}
Entre as refer�ncias utilizadas podemos citar:
\begin{itemize}
\item UML: \cite{prog-UML-blaha,prog-UML-Rumbaugh}.
\item Projetos: \cite{projetos-introducao,projetos-pmi,projetos-projetoeletrico1,projetos-Woiler}.
\item Gest�o de Projetos: \cite{projetos-Abrantes,projetos-gestao,projetos-Heldman,projetos-Menezes,projetos-Pahl,projetos-Valeriano,projetos-Rosa}
\item Produtos: \cite{projetos-Abrantes}.
\item C++: \cite{adb-prog-livroCpp}
\end{itemize}




\chapter{Etapas, Cronograma e Or�amento\label{cha: Etapas, Cronograma e Or=0000E7amento}}

\lhead{\thechapter\space - Atividades, Cronograma e Or�amento}\pagenumbering{arabic}

Neste cap�tulo temos a lista das etapas, das atividades o cronograma
e or�amento.

\section{Etapas\label{sec: Etapas}}

Esta proposta, caso aprovada, ser� desenvolvida seguindo as etapas
abaixo descritas. Um detalhamento das etapas esta dispon�vel \href{https://github.com/ldsc/ProjetoEngenharia-0-Metodologia-Instrucoes-Etapas-ModeloProfessorBueno}{aqui}:
\begin{itemize}
\item Etapa 0 - Defini��o do desafio tecnol�gico
\begin{itemize}
\item Identifica��o do problema, descri��o do desafio tecnol�gico, objetivo
geral e solu��o esperada (TRL/CRL).
\end{itemize}
\item Etapa 1 - Elabora��o do pr�-projeto
\begin{itemize}
\item Elabora��o da primeira vers�o do projeto, rascunho inicial.
\end{itemize}
\item Etapa 2 - Elabora��o do projeto - detalhamento e contrato (este documento)
\begin{itemize}
\item Ap�s aprova��o do pr�-projeto os alunos detalham o mesmo gerando o
projeto. Os dados do pr�-projeto podem ser copiados para o projeto,
a seguir tudo deve ser detalhado. As etapas associadas devem ser bem
definidas e suas atividades (note que para cada etapa teremos diversas
atividades). O cronograma simplificado deve ter as etapas e prazos,
deve ser realista. O or�amento deve ser bem realizado.
\end{itemize}
\item Etapa 3 - Modelagem de engenharia: 
\begin{itemize}
\item Concep��o; Elabora��o; An�lise Orientada a Objeto; Projeto do Sistema;
Projeto Orientado a Objeto.
\item Realiza��o de testes l�gicos; Documenta��o (gera��o dos documentos
de modelagem e diagramas associados).
\item Voc� encontra diagramas associados a esta etapa nos ap�ndices e \href{https://sites.google.com/view/professorandreduartebueno/ensino/introdu\%25C3\%25A7\%25C3\%25A3o-ao-projeto-de-engenharia?authuser=0\#h.jzh3wtardi2q}{no site da disciplina (etapa modelagem)}.
\end{itemize}
\item Etapa 4 - Ciclos de planejamento, detalhamento e constru��o/implementa��o:

\begin{itemize}
\item Esta etapa � dividida em ciclos de planejamento, detalhamento e constru��o/implementa��o,
pense nisso como uma divis�o temporal do trabalho e das entregas dos
sub-produtos (partes do produto final).
\item Nesta etapa, a mais demorada do projeto, temos:
\begin{itemize}
\item Reuni�o no in�cio do ciclo:
\begin{itemize}
\item A equipe faz uma reuni�o onde selecionam as \emph{features/funcionalidades
}a serem implementadas.
\end{itemize}
\item Reuni�es di�rias:
\begin{itemize}
\item Fazem reuni�es di�rias, de 15 minutos, onde falam o que fizeram no
dia anterior e o que ir�o fazer no dia.
\item Na sexta-feira fazem um balan�o do que foi feito.
\end{itemize}
\item Reuni�o no final do ciclo
\begin{itemize}
\item No final do ciclo entregam o sub-produto que deve estar funcional.
\end{itemize}
\end{itemize}
\item O �ltimo ciclo � o ciclo de fechamento, o mesmo inclui:
\begin{itemize}
\item Realiza��o de testes de integra��o e confec��o dos manuais do desenvolvedor
e do usu�rio.
\item Os manuais devem ter v�rios testes do sistema rodando (com todas as
informa�\~{e}os para executar/testar o produto desenvolvido).
\end{itemize}
\item Voc� encontra diagramas associados a esta etapa nos ap�ndices e \href{https://sites.google.com/view/professorandreduartebueno/ensino/introdu\%25C3\%25A7\%25C3\%25A3o-ao-projeto-de-engenharia?authuser=0\#h.mwce3hjuxu1n}{no site da disciplina (etapa planejamento/detalhamento/constru��o)}.
\end{itemize}
\item Etapa 5 - Entrega do produto: 
\begin{itemize}
\item Verifica��es finais na documenta��o e testes.
\item Apresenta��o do produto.
\item Entrega do produto.
\end{itemize}
\end{itemize}

\section{Cronograma\label{sec:Cronograma}}

Apresenta-se a seguir o cronograma de execu��o do projeto.
\begin{itemize}
\item Exemplo considerando produto desenvolvido em 4-6 meses:
\end{itemize}
\begin{tabular}{|c|c|c|c|c|c|c|}
\hline 
M�s & 1 & 2 & 3 & 4 & 5 & 6\tabularnewline
\hline 
\hline 
Etapa 0 - Defini��o do desafio tecnol�gico & X &  &  &  &  & \tabularnewline
\hline 
Etapa 1 - Elabora��o do pr�-projeto & X & X &  &  &  & \tabularnewline
\hline 
Etapa 2 - Elabora��o do projeto - detalhamento e contrato &  & X & X &  &  & \tabularnewline
\hline 
Etapa 3 - Modelagem de engenharia &  &  & X &  &  & \tabularnewline
\hline 
Etapa 4 - Ciclos de planejamento, detalhamento e constru��o/implementa��o &  &  & c1 & c2 & c3 & F\tabularnewline
\hline 
Etapa 5 - Entrega do produto (defesa) &  &  &  &  &  & D\tabularnewline
\hline 
\end{tabular}
\begin{itemize}
\item Exemplo considerando produto desenvolvido em 12 meses:
\begin{itemize}
\item Note que estamos considerando 3 ciclos de desenvolvimento e para vers�o
final um prazo para finalizar detalhes e manuais.
\end{itemize}
\end{itemize}
\begin{tabular}{|c|c|c|c|c|c|c|c|c|c|c|c|c|}
\hline 
M�s & 1 & 2 & 3 & 4 & 5 & 6 & 7 & 8 & 9 & 10 & 11 & 12\tabularnewline
\hline 
\hline 
Etapa 0 - Desafio tecnol�gico & X &  &  &  &  &  &  &  &  &  &  & \tabularnewline
\hline 
Etapa 1 - Pr�-projeto & X &  &  &  &  &  &  &  &  &  &  & \tabularnewline
\hline 
Etapa 2 - Projeto  & X & X &  &  &  &  &  &  &  &  &  & \tabularnewline
\hline 
Etapa 3 - Modelagem  &  & X & X &  &  &  &  &  &  &  &  & \tabularnewline
\hline 
Etapa 4 - Ciclos  &  &  & c1 & c1 & c1 & c2 & c2 & c2 & c3 & c3 & c3 & F\tabularnewline
\hline 
Etapa 5 - Entrega do produto (defesa) &  &  &  & c1 &  &  & c2 &  &  & c3 &  & D\tabularnewline
\hline 
\end{tabular}
\begin{itemize}
\item Exemplo considerando produto desenvolvido em 18 meses:
\begin{itemize}
\item Note que estamos considerando 3 ciclos de desenvolvimento e para vers�o
final um prazo para finalizar detalhes e manuais.
\end{itemize}
\end{itemize}
\begin{tabular}{|c|c|c|c|c|c|c|c|c|c|c|c|c|c|c|c|c|c|}
\hline 
M�s corrido &  & 1 & 2 & 3 & 4 & 5 & 6 & 7 & 8 & 9 & 10 & 11 & 12 & 13 & 14 & 15 & 16\tabularnewline
\hline 
M�s do ano & 2 & \textbf{3} & \textbf{4} & \textbf{5} & \textbf{6} & 7 & \textbf{8} & \textbf{9} & \textbf{10} & \textbf{11} & \textbf{12} & 1 & 2 & \textbf{3} & \textbf{4} & \textbf{5} & \textbf{6}\tabularnewline
\hline 
\hline 
Etapa 0 - Desafio & X &  &  &  &  &  &  &  &  &  &  &  &  &  &  &  & \tabularnewline
\hline 
Etapa 1 - Pr�-projeto &  & X &  &  &  &  &  &  &  &  &  &  &  &  &  &  & \tabularnewline
\hline 
Etapa 2 - Projeto &  & X & X &  &  &  &  &  &  &  &  &  &  &  &  &  & \tabularnewline
\hline 
Etapa 3 - Modelagem &  &  & X & X &  &  &  &  &  &  &  &  &  &  &  &  & \tabularnewline
\hline 
Etapa 4 - Ciclos &  &  &  &  & c1 & f & c1 & c1 & c2 & c2 & c2 & f & c3 & c3 & c3 & c3 & F\tabularnewline
\hline 
Etapa 5 - Entrega &  &  &  & M &  &  &  & e1 &  &  & e2 &  &  &  &  & e3 & D\tabularnewline
\hline 
\end{tabular}
\begin{itemize}
\item Exemplo considerando produto desenvolvido em 24 meses:
\item Note que estamos considerando 4 ciclos de desenvolvimento e para vers�o
final um prazo para finalizar detalhes e manuais.
\end{itemize}
\begin{tabular}{|c|c|c|c|c|c|c|c|c|c|c|c|c|c|}
\hline 
M�s corrido & 1 & 3 & 5 & 7 & 9 & 11 & 13 & 15 & 17 & 19 & 21 & 23 & 24\tabularnewline
\hline 
M�s do ano & 3 & 5 & 7 & 9 & 11 & 1 & 3 & 5 & 7 & 9 & 11 &  & \tabularnewline
\hline 
\hline 
Etapa 0 - Desafio & X &  &  &  &  &  &  &  &  &  &  &  & \tabularnewline
\hline 
Etapa 1 - Pr�-projeto & X &  &  &  &  &  &  &  &  &  &  &  & \tabularnewline
\hline 
Etapa 2 - Projeto & X & X &  &  &  &  &  &  &  &  &  &  & \tabularnewline
\hline 
Etapa 3 - Modelagem &  & X & X &  &  &  &  &  &  &  &  &  & \tabularnewline
\hline 
Etapa 4 - Ciclos &  &  & c1 & c1 & c1 & c2 & c2 & c2 & c3 & c3 & c3 & F & c3\tabularnewline
\hline 
Etapa 5 - Entrega &  &  &  & c1 &  &  & c2 &  &  & c3 &  & D & \tabularnewline
\hline 
\end{tabular}

\section{Or�amento\label{sec:Or=0000E7amento} }

Colocar aqui as informa��es sobre o or�amento do projeto.
\begin{itemize}
\item Considerar os equipamentos (custo e deprecia��o associada);
\item Considerar a m�o de obra;
\item Considerar outros custos (pessoal, infraestrutura, administrativos,
judici�rios, etc);
\end{itemize}

\section{Informa��es Extras Mecanismos Gest�o\label{sec:  Informa=0000E7=0000F5es Extras Mecanismos Gest=0000E3o}}
\begin{itemize}
\item Coloque aqui informa��es extras pertinentes.
\end{itemize}



\noindent\lhead{Refer�ncias} 
\chead{} 
\rhead{}Etapas-Padr�es/Programadas\bibliographystyle{apalike}
\bibliography{bibliografia}

\printindex
\end{document}
